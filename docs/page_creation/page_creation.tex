\documentclass[hidelinks,12pt,a4paper,numbers=enddot]{scrartcl}

\usepackage[margin=2cm,bottom=3cm]{geometry}
\usepackage{hyperref}
\usepackage{listings}
\usepackage{xcolor}
\usepackage{lmodern}

\title{How to make a content page with Jekyll}
\author{Joey Bevilacqua}

% listings configuration
\lstset{
	basicstyle=\small\ttfamily,
	frame=shadowbox,
	rulesepcolor=\color{gray},
	columns=fullflexible,
	commentstyle=\color{gray},
	keywordstyle=\bfseries\color{red},
	escapeinside={\%*}{*)},
  aboveskip=2em,
  captionpos=b,
  abovecaptionskip=1em,
  belowcaptionskip=1em
}

\begin{document}

\maketitle
\tableofcontents
\listoffigures
\newpage

\section{Introduction}
We are using Jekyll to build our website: we've told you this allows us to split
styling from contents, but what does this actually mean?
The CSS team has worked to provide you templates to build your page upon.
Using a template is easy once you get familiar with \textit{Jekyll}.

\section{The template header}

The template header consist in a series of variables defined at the very beginning
of the web page. These variables have values associated that will be used
to generate a proper styling.

\paragraph{}
This is an example for a page with the following stuff:

\begin{table}[h]
\centering{
\begin{tabular}{l l p{9cm}}
\textbf{Variable} & \textbf{Example value} & \textbf{Explaination}\\\hline
\texttt{layout} & page & The general style of the page. Use "page"\\
\texttt{topic} & The \textbf{ls} command & What the page is about\\
\texttt{author} & Claudio Maggioni & The main author of the page\\
\texttt{category\_title} & Basic commands & Name of the category (Your team full name)\\
\texttt{category-page} & basic & The name of the directory this file is in\\
\texttt{tags} & list file directory find & List of words separated by spaces that are relevant to this (Think of these as hashtags without the \#)\\
\end{tabular}
}
\end{table}

\begin{figure}[h]
\begin{lstlisting}[language=html]
---
layout: page
category_title: Basic commands
category-page: basic
tags: directory list
author: Claudio Maggioni
title: ls
---
\end{lstlisting}
\caption{An example of a template header}
\end{figure}

\section{Content}

\paragraph{}
The content comes right after the template header.
Keep your content as simple as possible, also don't worry about having a long page.

\subsection{Don'ts}
\textbf{DO NOT} insert screenshots of your terminal or any other kind of screenshot
unless you are a member of the Basic team trying to show how to open the shell. Use
the code tags and eventually describe the results using text.

\paragraph{}
\textbf{DO NOT} use \texttt{div} and \texttt{span}s, nor use \texttt{class=}, \texttt{id=}
or other styling attributes or options before talking with the CSS team. 

\subsection{File placement}
Place the \texttt{html} files straight inside the team folder: the creation of subfolders is not
allowed. Name the \texttt{html} file using the command or topic you're talking about name.

\begin{figure}[h]
\begin{lstlisting}[language=html]
<p>The <code>ls</code> command is used to list a directory content or a file.<br>
The name stands for <i>LiSt</i>.

<h2>Usage</h2>

<p>The default ls command syntax is:

<pre>
ls [flags] [path]
</pre>

Where [flags] are the ls flags, read below for more info,
and [path] is the (optional) path (absolute or relative).
If no path is provided the current directory is listed.
</p>
 
<h2>Flags<h2>

<p>Here are some of the most common ls flags:

<ul>
  <li> <b>-a</b>: Includes the hidden files (those which name starts with ".") </li>
  <li> <b>-l</b>: Lists with more information about the list file(s) </li>
  <li> %*\ldots*) </li>
</ul>

</p>
\end{lstlisting}
\caption{An example for the content of a page}
\end{figure}

\end{document}